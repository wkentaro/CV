% !TEX TS-program = xelatex
% !TEX encoding = UTF-8 Unicode
% -*- coding: UTF-8; -*-
% vim: set fenc=utf-8

%%%%%%%%%%%%%%%%%%%%%%%%%%%%%%%%%%%%%%%%%%%%%%%%%%%%%%%%%%%%%%%%%
%% SIMPLE-RESUME-CV
%% <https://github.com/zachscrivena/simple-resume-cv>
%% This is free and unencumbered software released into the
%% public domain; see <http://unlicense.org> for details.
%%%%%%%%%%%%%%%%%%%%%%%%%%%%%%%%%%%%%%%%%%%%%%%%%%%%%%%%%%%%%%%%%

%%%%%%%%%%%%%%%%%%%%%%%%%%%%%%%%%%%%%%%%%%%%%%%%%%%%%%%%%%%%%%%%%
%% INSTRUCTIONS FOR COMPILING THIS DOCUMENT ("CV.tex")
%% TeX ---(XeLaTeX)---> PDF:
%%
%% Method 1: Use latexmk for fully automated document generation:
%%   latexmk -xelatex "CV.tex"
%%   (add the -pvc switch to automatically recompile on changes)
%%
%% Method 2: Use XeLaTeX directly:
%%   xelatex "CV.tex"
%%   (run multiple times to resolve cross-references if needed)
%%%%%%%%%%%%%%%%%%%%%%%%%%%%%%%%%%%%%%%%%%%%%%%%%%%%%%%%%%%%%%%%%

\documentclass[letterpaper,MMMyyyy,nonstop]{simpleresumecv}
% Class options:
% a4paper, letterpaper, draft, nonstop
% MMMyyyy, ddMMMyyyy, MMMMyyyy, ddMMMMyyyy, yyyyMMdd, yyyyMM, yyyy

%%%%%%%%%%%%%%%%%%%%%%%%%%%%%%%%%%%%%%%%%%%%%%%%%%%%%%%%%%%%%%%%%
%% PREAMBLE.
%%%%%%%%%%%%%%%%%%%%%%%%%%%%%%%%%%%%%%%%%%%%%%%%%%%%%%%%%%%%%%%%%

% CV Info (to be customized).
\newcommand{\CVAuthor}{Kentaro Wada}
\newcommand{\CVTitle}{Kentaro Wada's CV}
\newcommand{\CVNote}{CV compiled on {\today}}
\newcommand{\CVWebpage}{wkentaro.com}

% PDF settings and properties.
\hypersetup{
pdftitle={\CVTitle},
pdfauthor={\CVAuthor},
pdfsubject={\CVWebpage},
pdfcreator={XeLaTeX},
pdfproducer={},
pdfkeywords={},
pdfpagemode={},
unicode=true,
bookmarks=true,
bookmarksopen=true,
pdfstartview=FitH,
pdfpagelayout=OneColumn,
pdfpagemode=UseOutlines,
hidelinks,
breaklinks}

% Shorthand.
\newcommand{\CodeCommand}[1]{\mbox{\textbf{\textbackslash{#1}}}}

%%%%%%%%%%%%%%%%%%%%%%%%%%%%%%%%%%%%%%%%%%%%%%%%%%%%%%%%%%%%%%%%%
%% ACTUAL DOCUMENT.
%%%%%%%%%%%%%%%%%%%%%%%%%%%%%%%%%%%%%%%%%%%%%%%%%%%%%%%%%%%%%%%%%

\begin{document}

%%%%%%%%%%%%%%%
% TITLE BLOCK %
%%%%%%%%%%%%%%%

\title{\CVAuthor}

\begin{subtitle}
5-14-24 Sendagi, Bunkyo-ku, Tokyo, 1130022, Japan
\par
www.kentaro.wada@gmail.com
\,\SubBulletSymbol\,
+81\,(80)\,6177-5221
\,\SubBulletSymbol\,
\CVWebpage
\par
Date of birth: 31 January 1994
\,\SubBulletSymbol\,
Nationality: Japanese

\noindent\makebox[\linewidth]{\rule{0.8\paperwidth}{0.4pt}}
\end{subtitle}

\begin{body}

%%%%%%%%%%%%%%%
%% EDUCATION %%
%%%%%%%%%%%%%%%

\section
{Education}
{Education}
{PDF:Education}

\textbf{University of Tokyo}
\newline
MS in Information Science and Technology
\hfill
{\it \DatestampYM{2016}{09} -- \DatestampYM{2018}{08} (expected)}
\newline
BE in Mechano-Informatics
\hfill
{\it \DatestampYM{2012}{04} -- \DatestampYM{2016}{03}}
\newline
Supervisors: Prof. Masayuki Inaba, Prof. Kei Okada

%%%%%%%%%%%%%%%
%% PORTFOLIO %%
%%%%%%%%%%%%%%%

\section
{Portfolio}
{Portfolio}
{PDF:Portfolio}

\CVWebpage
\newline
{\it Extensive listing of cocurricular and research projects.}

%%%%%%%%%%%%%%%%%
%% DISTINCTION %%
%%%%%%%%%%%%%%%%%

\section
{Distinction}
{Distinction}
{PDF:Distinction}

University of Tokyo, Toyota Dwango Advanced AI Fellowship
\hfill
{\it \DatestampY{2017}}

\BigGapNoBreak

Google Summer of Code Student
\hfill
{\it \DatestampY{2016}}
\newline
{\it Completed an open source project from the Open Source Robotics Foundation.}

\BigGapNoBreak

5th Place Winners (Pick Task) at the Amazon Picking Challenge
\hfill
{\it \DatestampY{2016}}
\newline
{\it An internationally recognised premier robotics competition.}

%%%%%%%%%%%%%%%%%%
%% PUBLICATIONS %%
%%%%%%%%%%%%%%%%%%

\section
{Publications}
{Publications}
{PDF:Publications}

% \subsection
% {Journals}
% {Journals}
% {PDF:Journals}

% \BigGap
% \subsection
% {International Conferences}
% {International Conferences}
% {PDF:InternationalConferences}

% Note the use of {\CharSpace} for aligning shorter numbers.
% \NumberedItem{{\CharSpace}[14]}
% \href{http://www.icra2018.org/}
\textbf{Kentaro Wada}, Shun Hasegawa, Shingo Kitagawa, Yuto Uchimi, Naoya Yamaguchi, Kei Okada, and Masayuki Inaba,
``Few-shot Learning Based on Context-aware Network Expansion with Artificial Training Data for Picking in Warehouse Automation'',
\textit{Under review at the IEEE International Conference on Robotics and Automation (ICRA)}, 2018.
\href{https://drive.google.com/open?id=1wLL3vXzOuxBeimeHqjvpc74gMmeipoCJ}{\underline{[Paper]}}
\href{https://drive.google.com/open?id=1-hWOkueOqEYKIV0WDsxO3JOroIE56F8u}{\underline{[Movie]}}

\BigGapNoBreak

% Note the use of {\CharSpace} for aligning shorter numbers.
% \NumberedItem{{\CharSpace}[9]}
% \href{http://www.iros2017.org/}
\textbf{Kentaro Wada}, Kei Okada, and Masayuki Inaba,
``Probabilistic 3D Multilabel Real-time Mapping for Multi-object Manipulation'',
\textit{IEEE/RSJ International Conference on Intelligent Robots and Systems (IROS)}, 2017.
\href{https://drive.google.com/open?id=1frqieyHiQBqpr1e9mWPrzfaX8FbpLcuX}{\underline{[Paper]}}
\href{https://drive.google.com/open?id=1nUFB1_jHLomxAEhlWsgaMRkdGycB5Dhk}{\underline{[Movie]}}
.
% Vancouver, Canada. 2017.
% \DatestampYM{2017}{9}.

\BigGapNoBreak

% Note the use of {\CharSpace} for aligning shorter numbers.
% \NumberedItem{{\CharSpace}[8]}
% \href{http://www.iros2017.org/}
Shun Hasegawa, \textbf{Kentaro Wada}, Yusuke Niitani, Kei Okada, and Masayuki Inaba,
``A Three-Fingered Hand with a Suction Gripping System for Picking Various Objects in Cluttered Narrow Space'',
\textit{IEEE/RSJ International Conference on Intelligent Robots and Systems (IROS)}, 2017.
\href{https://drive.google.com/open?id=1JK_o178iRMlQUQQgIlBLlltsj7cwMDeE}{\underline{[Paper]}}
\href{https://drive.google.com/open?id=1JPzAq4B47cjH35bhAjGRFMXyUN7voO4o}{\underline{[Movie]}}
% Vancouver, Canada.
% \DatestampYM{2017}{9}.

\BigGapNoBreak

% Note the use of {\CharSpace} for aligning shorter numbers.
% \NumberedItem{{\CharSpace}[7]}
\textbf{Kentaro Wada}, Makoto Sugiura, Iori Yanokura, Yuto Inagaki, Kei Okada, and Masayuki Inaba,
``Pick-and-Verify: Verification-based Highly Reliable Picking System for Various Target Objects in Clutter'',
\textit{Journal of Advanced Robotics}, 2017.
\href{http://www.tandfonline.com/doi/abs/10.1080/01691864.2016.1269672?journalCode=tadr20}{\underline{[Paper]}}
\href{http://www.tandfonline.com/doi/suppl/10.1080/01691864.2016.1269672?scroll=top}{\underline{[Movie]}}
% \DatestampYM{2017}{3}.

\BigGapNoBreak

% Note the use of {\CharSpace} for aligning shorter numbers.
% \NumberedItem{{\CharSpace}[6]}
% \href{https://ras.papercept.net/conferences/conferences/ICHR16/program/ICHR16_ContentListWeb_2.html}
\textbf{Kentaro Wada}, Masaki Murooka, Kei Okada, and Masayuki Inaba,
``3D Object Segmentation for Shelf Bin Picking by Humanoid with Deep Learning and Occupancy Voxel Grid Map'',
\textit{IEEE-RAS International Conference on Humanoid Robotics (Humanoids)}, 2016.
\href{http://ieeexplore.ieee.org/document/7803415/}{\underline{[Paper]}}
\href{https://drive.google.com/open?id=19YwO1LXAcRCfS8rAM5paxc2SEzGE3odA}{\underline{[Movie]}}
% Cancun, Mexico.
% \DatestampYM{2016}{11}.

\BigGapNoBreak

% Note the use of {\CharSpace} for aligning shorter numbers.
% \NumberedItem{{\CharSpace}[5]}
% \href{https://ras.papercept.net/conferences/conferences/ICHR16/program/ICHR16_ContentListWeb_1.html}
Yuki Furuta, \textbf{Kentaro Wada}, Masaki Murooka, Shunichi Nozawa, Yohei Kakichi, Kei Okada and Masayuki Inaba,
``Transformable Semantic Map Based Navigation Using Autonomous Deep Learning Object Segmentation'',
\textit{IEEE-RAS International Conference on Humanoid Robotics (Humanoids)}, 2016.
\href{http://ieeexplore.ieee.org/document/7803338/}{\underline{[Paper]}}
\href{https://drive.google.com/open?id=1ljJQ4SUjkNT1S1GWb4h7h__zbStz3oOo}{\underline{[Movie]}}
% Cancun, Mexico.
% \DatestampYM{2016}{11}.

% \BigGap
% \subsection
% {Domestic Conferences}
% {Domestic Conferences}
% {PDF:DomesticConferences}
%
% \GapNoBreak
% \NumberedItem{{\CharSpace}[10]}
% % \href{https://www.ai-gakkai.or.jp/jsai2017}
% \textbf{K.~Wada}, K.~Okada and M.~Inaba,
% ``Fully Convolutional Object Depth Prediction for 3D Segmentation from 2.5D Input'',
% in \textit{Annual Conference of the Japanese Society for Artificial Intelligence 2017},
% Aichi, Japan.
% \DatestampYM{2017}{5}.
%
% \GapNoBreak
% \NumberedItem{{\CharSpace}[11]}
% % \href{https://www.ai-gakkai.or.jp/jsai2017}
% M.~Murooka, Y.~Niitani, \textbf{K.~Wada}, S.~Nozawa, Y.~Kakiuchi, K.~Okada and M.~Inaba,
% ``Motion Prediction of Object in Image by Deep Learning for Robot Manipulation'',
% in \textit{Annual Conference of the Japanese Society for Artificial Intelligence 2017},
% Aichi, Japan.
% \DatestampYM{2017}{5}.
%
% \GapNoBreak
% \NumberedItem{{\CharSpace}[12]}
% % \href{https://www.ai-gakkai.or.jp/jsai2017}
% S.~Kitagawa, \textbf{K.~Wada}, K.~Okada and M.~Inaba,
% ``Learning-based Task Failure Prediction and Selective Execution of Dual-arm Support Motion for Stowing Task'',
% in \textit{Annual Conference of the Japanese Society for Artificial Intelligence 2017},
% Aichi, Japan.
% \DatestampYM{2017}{5}.
%
% \GapNoBreak
% \NumberedItem{{\CharSpace}[13]}
% % \href{http://robomech.org/2017/en/}
% S.~Hasegawa, \textbf{K.~Wada}, K.~Okada and M.~Inaba,
% ``Development of Suction Pinching Hand for Picking Task in Narrow Space'',
% in \textit{2017 JSME Conference on Robotics and Mechatronics},
% Fukushima, Japan.
% \DatestampYM{2017}{5}.
%
% \GapNoBreak
% % Note the use of {\CharSpace} for aligning shorter numbers.
% \NumberedItem{{\CharSpace}[4]}
% % \href{http://rsj2016.rsj-web.org/}
% Y.~Niitani, \textbf{K.~Wada}, S.~Hasegawa, S.~Kitagawa, M.~Bando, K~Okada, and M~Inaba,
% ``Semantic Image Segmentation and 3D Object Outline Extraction with Deep Learning for Picking Objects from Shelf-bin'',
% in \textit{Annual Conference of The Robotics Society of Japan},
% Yamagata, Japan.
% \DatestampYM{2016}{9}.
%
% \GapNoBreak
% % Note the use of {\CharSpace} for aligning shorter numbers.
% \NumberedItem{{\CharSpace}[3]}
% % \href{http://robomech.org/2016/en/}
% \textbf{K.~Wada}, K.~Okada and M.~Inaba,
% ``Advanced Multi-layered Perception for Picking in Clutter with Parameter Reinforcement Learning via Experiment in Task'' (in Japanese),
% in \textit{The Robotics and Mechatronics Conference 2016},
% Kanagawa, Japan.
% \DatestampYM{2016}{6}.
%
% \GapNoBreak
% % Note the use of {\CharSpace} for aligning shorter numbers.
% \NumberedItem{{\CharSpace}[2]}
% % \href{http://rsj2015.rsj-web.org/}
% \textbf{K.~Wada}, I.~Yanokura, M.~Sugiura, Y.~Inagaki, K.~Okada and M.~Inaba,
% ``Daily Object Picking System with Visual Verification and Vacuum Gripper on Dual-arm Robot'' (in Japanese),
% in \textit{Annual Conference of Robotics Society Japan 2015},
% Tokyo, Japan.
% \DatestampYM{2015}{3}.
%
% \GapNoBreak
% % Note the use of {\CharSpace} for aligning shorter numbers.
% \NumberedItem{{\CharSpace}[1]}
% % \href{http://www.sigkst.org/index.php?site_id=&page=\%C2\%E823\%B2\%F3\%B8\%A6\%B5\%E6\%B2\%F1}
% \textbf{K.~Wada}, K.~Kawakami, Y.~Honda, K.~Tanaka,
% ``Customer Clustering with Big Data Analysis of Purchase History'' (in Japanese),
% in \textit{Japanese Artificial Intellicence Conference, SIG-KST 23th},
% Tokyo, Japan.
% \DatestampYM{2014}{11}.

%%%%%%%%%%%%%%%%%%%%%%%%%
%% RESEARCH EXPERIENCE %%
%%%%%%%%%%%%%%%%%%%%%%%%%

\section
{Research Experience}
{Research Experience}
{PDF:ResearchExperience}

Leader of the UTokyo Team at the Amazon Robotics Challenge
\hfill
\textit{\DatestampY{2015} -- \DatestampY{2017}}
\newline
\textit{JSK Robotics Laboratory at the University of Tokyo}
\newline
Supervisor: Associate Prof. Kei Okada
\newline
Objectives: To develop a robust state-of-the-art robot picking system for warehouse automation.
2015 edition: Verification based robust picking system by in-hand recognition.
2016 edition: Deep learning based 3D semantic segmentation.
2017 edition: Few-shot deep learning of novel object segmentation using only instance images.

\BigGapNoBreak

Research Assistant at the UTokyo JSK Robotics Lab
\hfill
\textit{\DatestampY{2015} -- \DatestampY{2017}}
\newline
\textit{JSK Robotics Laboratory at University of Tokyo}
\newline
Supervisor: Associate Prof. Kei Okada
\newline
Objectives: To develop a system of continuous integration of a robotic system as a whole:
(1) Same software as a robotic system on simulation and real world.
(2) Enable motion testing by a simulator with dynamics.

\BigGapNoBreak

Research Assistant at the UTokyo Tanaka Kenji Lab
\hfill
\textit{\DatestampY{2014} -- \DatestampY{2015}}
\newline
\textit{Tanaka Kenji Laboratory at the University of Tokyo}
\newline
Supervisor: Associate Prof. Kenji Tanaka
\newline
Objectives: To analyse customer data of an e-commerce site and segment the users' tastes by clustering
user data according to page access and shopping.

% \href{http://www.jsk.t.u-tokyo.ac.jp/}
% {\textbf{JSK Robotics Laboratory}},
% The University of Tokyo
%
% \GapNoBreak
% \BulletItem
% Graduate Research Student, Computer Science Department
% \hfill
% \DatestampYMD{2016}{10}{01} --
% Present
% \begin{detail}
% \SubBulletItem
% Project:
% Study of Robotic Manipulation with Learning for Object Segmentation (Master Thesis)
% \SubBulletItem
% Supervisors:
% Prof.~Masayuki~Inaba and Associate~Prof.~Kei~Okada
% \SubBulletItem
% Focus:
% Deep Learning, 3D Vision, Robotic Manipulation
% \end{detail}
%
% \GapNoBreak
% \BulletItem
% Research Assistant
% \hfill
% \DatestampYMD{2015}{10}{01} --
% \DatestampYMD{2017}{03}{31}
% \begin{detail}
% \SubBulletItem
% Project:
% Picking General Objects with Verification-based Vision System
% \SubBulletItem
% Supervisors:
% Associate~Prof.~Kei~Okada
% \SubBulletItem
% Focus:
% Deep Learning, 3D Vision, Robotic Manipulation
% \end{detail}
%
% \GapNoBreak
% \BulletItem
% Undergraduate Research Student, Engineering Department
% \hfill
% \DatestampYMD{2015}{04}{01} --
% \DatestampYMD{2016}{03}{31}
% \begin{detail}
% \SubBulletItem
% Project:
% Learning for Picking through Experience of Verification-based Perception System (Bachelor Thesis)
% \SubBulletItem
% Supervisors:
% Prof.~Masayuki~Inaba and Associate~Prof.~Kei~Okada
% \SubBulletItem
% Focus:
% Deep Learning, 3D Vision, Robotic Manipulation
% \end{detail}
%
%
% \href{http://www.jsk.t.u-tokyo.ac.jp/}
% {\textbf{Tanaka Kenji Laboratory}},
% The University of Tokyo
%
% \GapNoBreak
% \BulletItem
% Research Assistant
% \hfill
% \DatestampYMD{2014}{05}{01} --
% \DatestampYMD{2015}{03}{31}
% \begin{detail}
% \SubBulletItem
% Project:
% Customer Clustering with Big Data Analysis of Purchase History
% \SubBulletItem
% Supervisors:
% Associate~Prof.~Kenji~Tanaka
% \SubBulletItem
% Focus:
% Machine Learning, Data Mining
% \end{detail}

%%%%%%%%%%%%%%%%%%%%%%%%%%%
%% AWARDS & SCHOLARSHIPS %%
%%%%%%%%%%%%%%%%%%%%%%%%%%%

% \section
% {Awards \&\newline
% Scholarships}
% {Awards \& Scholarships}
% {PDF:AwardsAndScholarships}
%
% \BulletItem
% Dean's List,
% Fall 2002 through Spring 2005,
% Science College
% \hfill
% \DatestampY{2002} --
% \DatestampY{2005}
% \begin{detail}
% \SubItem
% For attaining a semester GPA of at least 3.75.
% \end{detail}
%
% \Gap
% \BulletItem
% Undergraduate Researcher Award,
% Science College
% \hfill
% \DatestampYMD{2005}{05}{15}
% \begin{detail}
% \SubItem
% For outstanding scientific contributions in the fields of lasers and climate change.
% \end{detail}
%
% \Gap
% \BulletItem
% Chess Tournament,
% First Prize,
% Science College
% \hfill
% \DatestampYMD{2003}{03}{10}
% \begin{detail}
% \SubItem
% Awarded at the Tenth Annual Chess Tournament held during Open House.
% \end{detail}
%
% \Gap
% \BulletItem
% International Science Scholarship,
% \hfill
% \DatestampYMD{2001}{12}{10}
% \newline
% Global Science, Technology, Engineering, and Mathematics Foundation
% \begin{detail}
% \SubItem
% Full-tuition scholarship with stipend for undergraduate studies.
% One of 42 awardees in the world.
% \end{detail}

%%%%%%%%%%%%%%%%%%%%%%%%%%%%%%%%%%%%%%%%%%%%
%% PROFESSIONAL AFFILIATIONS & ACTIVITIES %%
%%%%%%%%%%%%%%%%%%%%%%%%%%%%%%%%%%%%%%%%%%%%

% \section
% {Professional Affiliations\newline
% \& Activities}
% {Professional Affiliations \& Activities}
% {PDF:ProfessionalAffiliationsActivities}

% \GapNoBreak
% \href{https://www.amazonrobotics.com/#/roboticschallenge}
% {\textbf{Amazon Robotics Challenge 2017}},
% Nagoya, Japan
% \GapNoBreak
% \BulletItem
% \underline{K.~Wada}, S.~Hasegawa, S.~Kitagawa, Y.~Uchimi, N.~Yamaguchi, K.~Okada and M.~Inaba
% \BulletItem
% 12th/13th place in 16 teams for pick/stow tasks.
% \BulletItem
% A core member the team composed of 5 students and 2 professors.
% \BulletItem
% Especially worked for object recognition.
% \hfill
% \DatestampYM{2017}{04} --
% \DatestampYM{2017}{07}
%
% \href{https://www.amazonrobotics.com/#/pickingchallenge}
% {\textbf{Amazon Picking Challenge 2016}},
% Leipzig, Germany
% \GapNoBreak
% \BulletItem
% \underline{K.~Wada}, S.~Hasegawa, S.~Kitagawa, Y.~Niitani, M.~Bando, K.~Okada and M.~Inaba
% \BulletItem
% 5th/8th place in 16 teams for pick/stow tasks.
% \BulletItem
% A core member of the team composed of 5 students and 2 professors.
% \BulletItem
% Especially worked for object recognition.
% \hfill
% \DatestampYM{2016}{04} --
% \DatestampYM{2016}{07}
%
% \href{https://summerofcode.withgoogle.com/archive/2016/projects/4547228978905088/}
% {\textbf{Google Summer of Code 2016}},
% Tokyo, Japan
% \GapNoBreak
% \BulletItem
% \underline{K.~Wada}, F.~Proctor, S.~Edwards
% \BulletItem
% Student, Passed the Final Evaluation
% \hfill
% \DatestampYM{2016}{05} --
% \DatestampYM{2016}{08}
%
% \href{https://www.amazonrobotics.com/#/pickingchallenge}
% {\textbf{Amazon Picking Challenge 2015}},
% Seattle, USA
% \GapNoBreak
% \BulletItem
% \underline{K.~Wada}, I.~Yanokura, M.~Sugiura, Y.~Inagaki, K.~Okada and M.~Inaba
% \BulletItem
% 8th place in 28 teams.
% \BulletItem
% A core member of the team composed of 4 students and 2 professors.
% \BulletItem
% Worked for object recognition and robotic manipulation.
% \hfill
% \DatestampYM{2014}{10} --
% \DatestampYM{2015}{05}

%%%%%%%%%%%%%%%%%%%%%%%
%% CAMPUS ACTIVITIES %%
%%%%%%%%%%%%%%%%%%%%%%%

% \section
% {Campus Activities}
% {Campus Activities}
% {PDF:CampusActivities}
%
% \href{http://www.example.com/my-club}
% {\textbf{First Volunteers Club}},
% First American University
%
% \GapNoBreak
% \BulletItem
% President
% \hfill
% \DatestampYMD{2006}{08}{15} --
% \DatestampYMD{2007}{08}{15}
% \begin{detail}
% \SubBulletItem
% Lorem ipsum dolor sit amet, consectetur adipiscing elit.
% \SubBulletItem
% Curabitur vitae laoreet velit, vel ultricies est. Nam nec elit ac ante facilisis ultrices.
% \SubBulletItem
% Integer sit amet turpis dolor. Lorem ipsum dolor sit amet, consectetur adipiscing elit. Nunc at orci eu leo vulputate finibus sed et sem.
% \SubBulletItem
% Suspendisse volutpat sapien et mi cursus, gravida ornare mauris sollicitudin.
% \end{detail}

%%%%%%%%%%%%%%%%%%%%%%%%%%%
%% WORK EXPERIENCE %%
%%%%%%%%%%%%%%%%%%%%%%%%%%%

\section
{Work Experience}
{Work Experience}
{PDF:WorkExperience}

% \href{http://www.donuts.ne.jp}
Donuts Co. Ltd., Tokyo
\hfill
\textit{\DatestampY{2013} -- \DatestampY{2014}}
\newline
\textit{Interned as a System Integrator}
% \begin{detail}
% \SubBulletItem
% Frontend of e-commerce site with HTML, CSS and Javascript.
% \SubBulletItem
% Posting system construction with PHP.
% \end{detail}

\BigGapNoBreak

% \href{http://www.honda-ri.com/}
Honda Research Institute, Tokyo
\hfill
\textit{\DatestampY{2014}}
\newline
\textit{Summer intern, Road scene understanding with deep learning}
% \DatestampYMD{2014}{09}{26}
% \DatestampYMD{2014}{08}{29} --
% \DatestampYMD{2014}{09}{26}
% \begin{detail}
% \SubBulletItem
% Road scene recognition with deep learning
% \end{detail}

% %%%%%%%%%%%%%%%
% %% LANGUAGES %%
% %%%%%%%%%%%%%%%
%
% \section
% {Languages}
% {Languages}
% {PDF:Languages}
%
% \BulletItem
% Japanese: Native language.
%
% \GapNoBreak
% \BulletItem
% English: Fluent (listening, speaking, reading, writing).
%
% \GapNoBreak
% \BulletItem
% Chinese: Basic (listening, speaking, reading, writing).

%%%%%%%%%%%%
%% SKILLS %%
%%%%%%%%%%%%

\section
{Key Skills}
{Key Skills}
{PDF:KeySkills}

\BulletItem
High-level programming skills, especially with Python and C++,
trained in the research use and contributions to open source projects at
\href{http://github.com/wkentaro}{\underline{GitHub}}.

\BulletItem
Experience and knowledge of constructing a large robot vision system integrating various kinds of hardware and software
with the Robot Operating System (ROS).

\BulletItem
Knowledge of deep learning implementation with the frameworks including, Chainer, PyTorch and Caffe,
and GPU computing using CUDA.

% Programming Languages: Python, C++, CUDA, C, BHTML, CSS, Javascript, PHP, Lisp
%
% \GapNoBreak
%
% Frameworks: Chainer, Caffe, scikit-learn, ROS, PCL, OpenCV, scikit-image, flask

%%%%%%%%%%%%%%%
%% INTERESTS %%
%%%%%%%%%%%%%%%

\section
{Interests}
{Interests}
{PDF:Interests}

Deep learning,
Scene understanding,
3D reconstruction,
Real-time vision system.
% Basketball,
% western art,
% piano.

%%%%%%%%%%%%%%%%
%% REFERENCES %%
%%%%%%%%%%%%%%%%

\section
{References}
{References}
{PDF:References}

\textbf{Prof. Masayuki Inaba}
\newline
Professor of the Graduate School of Information Technology and Science
\newline
University of Tokyo
\newline
73A1, Engineering Building NO. 2, 7-3-1, Hongo, Bunkyo-ku, Tokyo, 1138656, Japan
\newline
\href{mailto:inaba@jsk.imi.i.u-tokyo.ac.jp}
{inaba@jsk.imi.i.u-tokyo.ac.jp}
\,\SubBulletSymbol\,
+81\,(3)\,5841-7416

\BigGapNoBreak

\textbf{Prof. Kei Okada}
\newline
Professor of the Graduate School of Information Technology and Science
\newline
University of Tokyo
\newline
73A2, Engineering Building NO. 2, 7-3-1, Hongo, Bunkyo-ku, Tokyo, 1138656, Japan
\newline
\href{mailto:k-okada@jsk.imi.i.u-tokyo.ac.jp}
{k-okada@jsk.imi.i.u-tokyo.ac.jp}
\,\SubBulletSymbol\,
+81\,(3)\,5841-7416

\end{body}

%%%%%%%%%%%
% CV NOTE %
%%%%%%%%%%%

\UseNoteFont%
\null\hfill%
[\textit{\CVNote}]%
\hspace{2.0mm}\null

\end{document}
