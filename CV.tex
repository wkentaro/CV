% !TEX TS-program = xelatex
% !TEX encoding = UTF-8 Unicode
% -*- coding: UTF-8; -*-
% vim: set fenc=utf-8

%%%%%%%%%%%%%%%%%%%%%%%%%%%%%%%%%%%%%%%%%%%%%%%%%%%%%%%%%%%%%%%%%
%% SIMPLE-RESUME-CV
%% <https://github.com/zachscrivena/simple-resume-cv>
%% This is free and unencumbered software released into the
%% public domain; see <http://unlicense.org> for details.
%%%%%%%%%%%%%%%%%%%%%%%%%%%%%%%%%%%%%%%%%%%%%%%%%%%%%%%%%%%%%%%%%

%%%%%%%%%%%%%%%%%%%%%%%%%%%%%%%%%%%%%%%%%%%%%%%%%%%%%%%%%%%%%%%%%
%% INSTRUCTIONS FOR COMPILING THIS DOCUMENT ("CV.tex")
%% TeX ---(XeLaTeX)---> PDF:
%%
%% Method 1: Use latexmk for fully automated document generation:
%%   latexmk -xelatex "CV.tex"
%%   (add the -pvc switch to automatically recompile on changes)
%%
%% Method 2: Use XeLaTeX directly:
%%   xelatex "CV.tex"
%%   (run multiple times to resolve cross-references if needed)
%%%%%%%%%%%%%%%%%%%%%%%%%%%%%%%%%%%%%%%%%%%%%%%%%%%%%%%%%%%%%%%%%

\documentclass[letterpaper,MMMyyyy,nonstop]{simpleresumecv}
% Class options:
% a4paper, letterpaper, draft, nonstop
% MMMyyyy, ddMMMyyyy, MMMMyyyy, ddMMMMyyyy, yyyyMMdd, yyyyMM, yyyy

%%%%%%%%%%%%%%%%%%%%%%%%%%%%%%%%%%%%%%%%%%%%%%%%%%%%%%%%%%%%%%%%%
%% PREAMBLE.
%%%%%%%%%%%%%%%%%%%%%%%%%%%%%%%%%%%%%%%%%%%%%%%%%%%%%%%%%%%%%%%%%

% CV Info (to be customized).
\newcommand{\CVAuthor}{Kentaro Wada}
\newcommand{\CVTitle}{Kentaro Wada's CV}
\newcommand{\CVNote}{CV compiled on {\today}}
\newcommand{\CVWebpage}{\href{https://wkentaro.com}{\underline{https://wkentaro.com}}}
\renewcommand{\FooterText}{\null}

\definecolor{mygreen}{RGB}{26, 188, 156}

% PDF settings and properties.
\hypersetup{
pdftitle={\CVTitle},
pdfauthor={\CVAuthor},
pdfsubject={\CVWebpage},
pdfcreator={XeLaTeX},
pdfproducer={},
pdfkeywords={},
pdfpagemode={},
unicode=true,
bookmarks=true,
bookmarksopen=true,
pdfstartview=FitH,
pdfpagelayout=OneColumn,
pdfpagemode=UseOutlines,
colorlinks=true,
urlcolor=mygreen,
% hidelinks,
breaklinks
}

% Shorthand.
\newcommand{\CodeCommand}[1]{\mbox{\textbf{\textbackslash{#1}}}}

%%%%%%%%%%%%%%%%%%%%%%%%%%%%%%%%%%%%%%%%%%%%%%%%%%%%%%%%%%%%%%%%%
%% ACTUAL DOCUMENT.
%%%%%%%%%%%%%%%%%%%%%%%%%%%%%%%%%%%%%%%%%%%%%%%%%%%%%%%%%%%%%%%%%

\begin{document}

%%%%%%%%%%%%%%%
% TITLE BLOCK %
%%%%%%%%%%%%%%%

\title{\CVAuthor}

\begin{subtitle}
\CVWebpage
\par
www.kentaro.wada@gmail.com
\,\SubBulletSymbol\,
+44\,7535-753123
\par
Date of birth: 31 January 1994
\,\SubBulletSymbol\,
Nationality: Japan
\,\SubBulletSymbol\,
Location: London, UK

\noindent\makebox[\linewidth]{\rule{0.8\paperwidth}{0.4pt}}
\end{subtitle}

\begin{body}

%%%%%%%%%%%%%%%
%% EDUCATION %%
%%%%%%%%%%%%%%%

\section
{Education}
{Education}
{PDF:Education}

\textbf{Imperial College London}
\newline
PhD in Computing
\hfill
{\it \DatestampY{2018} -- \DatestampY{2022}}
\newline
Supervisor: Prof. Andrew J. Davision

\GapNoBreak

\textbf{The University of Tokyo}
\newline
MS in Information Science and Technology
\hfill
{\it \DatestampY{2016} -- \DatestampY{2018}}
\newline
BE in Mechano-Informatics
\hfill
{\it \DatestampY{2012} -- \DatestampY{2016}}
\newline
Supervisors: Prof. Masayuki Inaba, Prof. Kei Okada


%%%%%%%%%%%%%%%%%%%%%%%%%%%
%% WORK EXPERIENCE %%
%%%%%%%%%%%%%%%%%%%%%%%%%%%

\section
{Work Experience}
{Work Experience}
{PDF:WorkExperience}

\textbf{Corvus Robotics Inc.}, San Francisco
\hfill
\textit{\DatestampY{2020} -- \DatestampY{2021}}
\newline
Computer vision engineer for semantic segmentation (remote, part-time).

% \href{http://www.donuts.ne.jp}
\textbf{Donuts Co. Ltd.}, Tokyo
\hfill
\textit{\DatestampY{2013} -- \DatestampY{2014}}
\newline
Web system engineer (part-time).


%%%%%%%%%%%%%%%%%
%% DISTINCTION %%
%%%%%%%%%%%%%%%%%

\section
{Distinction}
{Distinction}
{PDF:Distinction}

\textit{Contributions to the Open Source Community}
\hfill
{\it \DatestampY{2015} - \DatestampY{2022}}
\newline
Created popular software with 1-8k stars and 500-1000 daily traffics (e.g., \href{https://github.com/wkentaro/labelme}{\underline{Labelme}}, \href{https://github.com/wkentaro/gdown}{\underline{Gdown}}).

\BigGapNoBreak

\textit{PhD President's Scholarship of Imperial College London}
\hfill
{\it \DatestampY{2018} -- \DatestampY{2022}}
\newline
One of the fifty PhD students for the full funded scholarship.
% \href{https://www.imperial.ac.uk/study/pg/fees-and-funding/scholarships/presidents-phd-scholarships/}{\underline{[Webpage]}}

\BigGapNoBreak

\textit{Two Patents on Object 6D Pose Estimation}
\hfill
{\it \DatestampY{2021}}
\newline
Invented methods for 3D object-level scene understanding using vision sensors.
% \href{https://patents.google.com/patent/WO2021198666A1/}{\underline{[Patent]}}
% \href{https://patents.google.com/patent/WO2021198665A1/}{\underline{[Patent]}}

\BigGapNoBreak

\textit{IEEE Robotics and Automation Society Japan Joint Chapter Young Award at IROS 2018}
\hfill
{\it \DatestampY{2018}}
\newline
One of the five Japanese students nominated with their conference papers.
% \href{https://www.ieee-jp.org/section/tokyo/chapter/RA-24/RASJPYoungAward_ICRA2018.html}{\underline{[Webpage]}}

\BigGapNoBreak

\textit{Lead the UTokyo Team at the Amazon Robotics Challenge}
\hfill
{\it \DatestampY{2015} -- \DatestampY{2017}}
\newline
Won the 5th place our of 16 teams in 2016. Mainly worked on the vision part.
% \href{https://github.com/start-jsk/jsk_apc}{\underline{[Code]}}

% Google Summer of Code
% \hfill
% {\it \DatestampY{2016}}
% \newline
% \textit{Built a reactive motion planning with the Open Source Robotics Foundation.}


%%%%%%%%%%%%%%%%%%
%% PUBLICATIONS %%
%%%%%%%%%%%%%%%%%%

\section
{Publications}
{Publications}
{PDF:Publications}

\BulletItem
\textit{Kentaro Wada}, Stephen James, and Andrew J. Davison,
``ReorientBot: Learning Object Reorientation for Specific-Posed Placement'',
IEEE International Conference on Robotics and Automation (ICRA), 2022.
\href{https://arxiv.org/abs/2202.11092}{\underline{[Paper]}}
\href{https://youtu.be/ahWN84sWWJU}{\underline{[Video]}}
\href{https://reorientbot.wkentaro.com}{\underline{[Webpage]}}

\BulletItem
\textit{Kentaro Wada}, Stephen James, and Andrew J. Davison,
``SafePicking: Learning Safe Object Extraction via Object-Level Mapping'',
IEEE International Conference on Robotics and Automation (ICRA), 2022.
\href{https://arxiv.org/abs/2202.05832}{\underline{[Paper]}}
\href{https://youtu.be/ejjqiBqRRKo}{\underline{[Video]}}
\href{https://safepicking.wkentaro.com}{\underline{[Webpage]}}

\BulletItem
\textit{Kentaro Wada}, Edgar Sucar, Stephen James, Daniel Lenton, and Andrew J.
Davison,
``MoreFusion: Multi-object Reasoning for 6D Pose Estimation from Volumetric
Fusion'',
IEEE Conference on Computer Vision and Pattern Recognition (CVPR), 2020.
\href{https://arxiv.org/abs/2004.04336}{\underline{[Paper]}}
\href{https://youtu.be/6oLUhuZL4ko}{\underline{[Video]}}
\href{https://morefusion.wkentaro.com}{\underline{[Webpage]}}

\BulletItem
\textit{Kentaro Wada}, Shingo Kitagawa, Kei Okada, and Masayuki Inaba,
``Instance Segmentation of Visible and Occluded Regions for Finding and Picking Target from a Pile of Objects'',
IEEE International Conference on Intelligent Robots and Systems (IROS), 2018.
\href{https://arxiv.org/abs/2001.07475}{\underline{[Paper]}}
\href{https://youtu.be/tNLtXb04i3w}{\underline{[Video]}}

\hfill
\href{https://scholar.google.com/citations?user=JoeSwcoAAAAJ&hl=ja}{\underline{See more…}}

% \BulletItem
% \textit{Kentaro Wada}, Kei Okada, and Masayuki Inaba,
% ``Probabilistic 3D Multilabel Real-time Mapping for Multi-object Manipulation'',
% IEEE/RSJ International Conference on Intelligent Robots and Systems (IROS), 2017.
% \href{https://arxiv.org/abs/2001.05752}{\underline{[Paper]}}
% \href{https://youtu.be/T-vtVQT9sgc}{\underline{[Video]}}


%%%%%%%%%%%%%%%%%%%%%%%%%
%% RESEARCH EXPERIENCE %%
%%%%%%%%%%%%%%%%%%%%%%%%%

% \section
% {Research Experience}
% {Research Experience}
% {PDF:ResearchExperience}

% -------------------------------------------------------------------------------------------------
% -------------------------------------------------------------------------------------------------

% \BigGapNoBreak

% -------------------------------------------------------------------------------------------------
% Research Assistant at the UTokyo JSK Robotics Lab
% \hfill
% \textit{\DatestampY{2015} -- \DatestampY{2017}}
% \newline
% \textit{JSK Robotics Laboratory at University of Tokyo}
% \BulletItem
% Objectives: To develop a system of continuous integration of a robotic system as a whole:
% (1) Same software as a robotic system on simulation and real world.
% (2) Enable motion testing by a simulator with dynamics.
% -------------------------------------------------------------------------------------------------

% \BigGapNoBreak

% -------------------------------------------------------------------------------------------------
% Research Assistant at the UTokyo Tanaka Kenji Lab
% \hfill
% \textit{\DatestampY{2014} -- \DatestampY{2015}}
% \newline
% \textit{Tanaka Kenji Laboratory at the University of Tokyo}
% \newline
% Supervisor: Associate Prof. Kenji Tanaka
% \newline
% Objectives: To analyse customer data of an e-commerce site and segment the users' tastes by clustering
% user data according to page access and shopping.
% -------------------------------------------------------------------------------------------------

% \href{http://www.jsk.t.u-tokyo.ac.jp/}
% {\textbf{JSK Robotics Laboratory}},
% The University of Tokyo
%
% \GapNoBreak
% \BulletItem
% Graduate Research Student, Computer Science Department
% \hfill
% \DatestampYMD{2016}{10}{01} --
% Present
% \begin{detail}
% \SubBulletItem
% Project:
% Study of Robotic Manipulation with Learning for Object Segmentation (Master Thesis)
% \SubBulletItem
% Supervisors:
% Prof.~Masayuki~Inaba and Associate~Prof.~Kei~Okada
% \SubBulletItem
% Focus:
% Deep Learning, 3D Vision, Robotic Manipulation
% \end{detail}
%
% \GapNoBreak
% \BulletItem
% Research Assistant
% \hfill
% \DatestampYMD{2015}{10}{01} --
% \DatestampYMD{2017}{03}{31}
% \begin{detail}
% \SubBulletItem
% Project:
% Picking General Objects with Verification-based Vision System
% \SubBulletItem
% Supervisors:
% Associate~Prof.~Kei~Okada
% \SubBulletItem
% Focus:
% Deep Learning, 3D Vision, Robotic Manipulation
% \end{detail}
%
% \GapNoBreak
% \BulletItem
% Undergraduate Research Student, Engineering Department
% \hfill
% \DatestampYMD{2015}{04}{01} --
% \DatestampYMD{2016}{03}{31}
% \begin{detail}
% \SubBulletItem
% Project:
% Learning for Picking through Experience of Verification-based Perception System (Bachelor Thesis)
% \SubBulletItem
% Supervisors:
% Prof.~Masayuki~Inaba and Associate~Prof.~Kei~Okada
% \SubBulletItem
% Focus:
% Deep Learning, 3D Vision, Robotic Manipulation
% \end{detail}
%
%
% \href{http://www.jsk.t.u-tokyo.ac.jp/}
% {\textbf{Tanaka Kenji Laboratory}},
% The University of Tokyo
%
% \GapNoBreak
% \BulletItem
% Research Assistant
% \hfill
% \DatestampYMD{2014}{05}{01} --
% \DatestampYMD{2015}{03}{31}
% \begin{detail}
% \SubBulletItem
% Project:
% Customer Clustering with Big Data Analysis of Purchase History
% \SubBulletItem
% Supervisors:
% Associate~Prof.~Kenji~Tanaka
% \SubBulletItem
% Focus:
% Machine Learning, Data Mining
% \end{detail}

%%%%%%%%%%%%%%%%%%%%%%%%%%%
%% AWARDS & SCHOLARSHIPS %%
%%%%%%%%%%%%%%%%%%%%%%%%%%%

% \section
% {Awards \&\newline
% Scholarships}
% {Awards \& Scholarships}
% {PDF:AwardsAndScholarships}
%
% \BulletItem
% Dean's List,
% Fall 2002 through Spring 2005,
% Science College
% \hfill
% \DatestampY{2002} --
% \DatestampY{2005}
% \begin{detail}
% \SubItem
% For attaining a semester GPA of at least 3.75.
% \end{detail}
%
% \Gap
% \BulletItem
% Undergraduate Researcher Award,
% Science College
% \hfill
% \DatestampYMD{2005}{05}{15}
% \begin{detail}
% \SubItem
% For outstanding scientific contributions in the fields of lasers and climate change.
% \end{detail}
%
% \Gap
% \BulletItem
% Chess Tournament,
% First Prize,
% Science College
% \hfill
% \DatestampYMD{2003}{03}{10}
% \begin{detail}
% \SubItem
% Awarded at the Tenth Annual Chess Tournament held during Open House.
% \end{detail}
%
% \Gap
% \BulletItem
% International Science Scholarship,
% \hfill
% \DatestampYMD{2001}{12}{10}
% \newline
% Global Science, Technology, Engineering, and Mathematics Foundation
% \begin{detail}
% \SubItem
% Full-tuition scholarship with stipend for undergraduate studies.
% One of 42 awardees in the world.
% \end{detail}

%%%%%%%%%%%%%%%%
%% KEY SKILLS %%
%%%%%%%%%%%%%%%%

\section
{Key Skills}
{Key Skills}
{PDF:KeySkills}

\BulletItem
\textbf{Coding and software development} with Python and C++ for GUI applications, command-line tools, deep learning, volumetric reconstruction, 2D/3D visualization, and physics simulation.

\GapNoBreak

\BulletItem
\textbf{SLAM for 3D semantic scene understanding} with expertise in object tracking, reconstruction, detection and pose estimation using onboard, moving vision sensors.

\GapNoBreak

\BulletItem
\textbf{Real-time vision and robotic system building} for 3D scene understanding and motion generation with expertise in integration using The Robot Operation System (ROS).

% Programming Languages: Python, C++, CUDA, C, BHTML, CSS, Javascript, PHP, Lisp
%
% \GapNoBreak
%
% Frameworks: Chainer, Caffe, scikit-learn, ROS, PCL, OpenCV, scikit-image, flask

\BigGapNoBreak
%%%%%%%%%%%%%%%
%% INTERESTS %%
%%%%%%%%%%%%%%%

\section
{Interests}
{Interests}
{PDF:Interests}

Deep Learning,
3D Computer Vision,
Robotics
% Basketball,
% western art,
% piano.

\end{body}

%%%%%%%%%%%
% CV NOTE %
%%%%%%%%%%%

% \UseNoteFont%
% \null\hfill%
% [\textit{\CVNote}]%
% \hspace{2.0mm}\null

\end{document}
